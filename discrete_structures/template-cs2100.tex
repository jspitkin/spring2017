% HW Template for CS 2100, taken from https://www.cs.cmu.edu/~ckingsf/class/02-714/hw-template.tex

% You don't need to use LaTeX or this template, but when learning LaTeX myself it helped to have a template to work from. I think it is worth the learning curve as you will use it throughout your academic career. Don't get overwhelmed at learning what all the template code does for now, learn as you go.%

\documentclass[11pt]{article}
\usepackage{amsmath,amssymb,amsthm}
\usepackage{graphicx}
\usepackage[margin=1in]{geometry}
\usepackage{fancyhdr}
\usepackage{algorithm}
\usepackage{algpseudocode}
\usepackage{pifont}
\setlength{\parindent}{0pt}
\setlength{\parskip}{5pt plus 1pt}
\setlength{\headheight}{13.6pt}
\newcommand\question[1]{\vspace{.25in}\hrule\textbf{#1}\vspace{.5em}\hrule\vspace{.10in}}
\renewcommand\part[1]{\Large\vspace{.10in}\textbf{#1}\vspace{.10in}\normalsize\newline}
\pagestyle{fancyplain}
\lhead{\textbf{\NAME\ (\UID)}}
\chead{\textbf{HW\HWNUM}}
\rhead{CS 6150, \today}
\begin{document}\raggedright

\newcommand\NAME{Jane Doe}  % REPLACE WITH YOUR NAME
\newcommand\UID{u0123456}     % REPLACE WITH YOUR UID
\newcommand\HWNUM{1}              % REPLACE WITH THE HOMEWORK NUMBER

% start deleting here

\textbf{\large{Delete these examples once you look over the commands and my comments in the .tex file.}}

\textit{Italic text} % '\textit{your text} will italicize your text.

\textbf{Bold text} % '\textbf{your text} will bold face your text.
% notice the blank space between lines to put them on each on their own line.

$c = \pi * r^2$ % wrapping text in '$' produces in-line math equations. There are special commands such as '\pi' that can only be used in math-mode (between $'s).

$$ (a \ \land b) \ \lor \ (\neg c \ \land \ d) $$ % wrapping text in '$$' produces math equations centered on a new line. The same math commands are used in both '$' and '$$'. Notice above I use '\ ' (a forward slash followed by a space) to produce a space character. I think this makes statements more readable. Also notice the command '\land' (logical and), '\lor' (logical or), and '\neg' (negation).

\framebox[1.5\width]{$4 + 5 = 9$} % wraps a math equation in a box. I think this is very useful as a way to display your final example (as I do in the sample problem). Note that the contents must be a math equation (wrapped in '$').


\begin{table}[H]
\centering
{\renewcommand{\arraystretch}{1.2}
\begin{tabular}{| c | c | c |} % The number of columns in your table, separated by '|'
\hline
p & q & $p \implies q$\\ % the header of your table, each value separated by '&'
\hline
T & T & T \\ \hline % Each row in your table, separated by '&'.
T & F & F \\ \hline % Keep the '\\ \hline' at the end of each row.
F & T & T \\ \hline % I suggest just copying and pasting the last table you made
F & F & T \\ \hline % each time you need a new table.
\end{tabular}}
\caption{The truth table for an implication.}
\end{table}

% stop deleting here

\question{Section 1.2} % this command is for each section in your homework

\part{1.n} 
% start deleting here
$T(n) = 4T(n/4) + n$

There will be $k$ levels of recursion where $k = log_4(n)$. The amount of work done at each level $k$ is summed. There is $T(1) * n$ work done by the bottom level.
$$\sum\limits_{i=1}^k 4^{i-1} * \frac{n}{4^{i-1}} + T(1) * n$$ 
$$\sum\limits_{i=1}^k n + T(1) * n$$
$$k * n + T(1) * n$$
$$log_4(n) * n + c * n$$

\framebox[1.5\width]{$O(n\log n)$}
% stop deleting here

\part{4.d} % this command is for each part or question. Leave a whitespace above each part.

\part{5.d}

\part{29.a}

\part{29.b}

\part{29.c}

\question{Section 1.3}

\end{document}