%%%%%%%%%%%%%%%%%%%%%%%%%%%%%%%%%%%%%%%%%
% Thin Sectioned Essay
% LaTeX Template
% Version 1.0 (3/8/13)
%
% This template has been downloaded from:
% http://www.LaTeXTemplates.com
%
% Original Author:
% Nicolas Diaz (nsdiaz@uc.cl) with extensive modifications by:
% Vel (vel@latextemplates.com)
%
% License:
% CC BY-NC-SA 3.0 (http://creativecommons.org/licenses/by-nc-sa/3.0/)
%
%%%%%%%%%%%%%%%%%%%%%%%%%%%%%%%%%%%%%%%%%

%----------------------------------------------------------------------------------------
%	PACKAGES AND OTHER DOCUMENT CONFIGURATIONS
%----------------------------------------------------------------------------------------

\documentclass[a4paper, 11pt]{article} % Font size (can be 10pt, 11pt or 12pt) and paper size (remove a4paper for US letter paper)

\usepackage[protrusion=true,expansion=true]{microtype} % Better typography
\usepackage{graphicx} % Required for including pictures
\usepackage{wrapfig} % Allows in-line images
\usepackage{float}
\usepackage{mathpazo} % Use the Palatino font
\usepackage[T1]{fontenc} % Required for accented characters
\linespread{1.05} % Change line spacing here, Palatino benefits from a slight increase by default

\makeatletter
\renewcommand\@biblabel[1]{\textbf{#1.}} % Change the square brackets for each bibliography item from '[1]' to '1.'
\renewcommand{\@listI}{\itemsep=0pt} % Reduce the space between items in the itemize and enumerate environments and the bibliography

\renewcommand{\maketitle}{ % Customize the title - do not edit title and author name here, see the TITLE block below
\begin{flushright} % Right align
{\LARGE\@title} % Increase the font size of the title

\vspace{50pt} % Some vertical space between the title and author name

{\large\@author} % Author name
\\\@date % Date

\vspace{40pt} % Some vertical space between the author block and abstract
\end{flushright}
}

%----------------------------------------------------------------------------------------
%	TITLE
%----------------------------------------------------------------------------------------

\title{\textbf{Basic System - Reading Comprehension on Short Passages for Question Answering}\\ % Title
} % Subtitle

\author{\textsc{Jake Pitkin} % Author
\\{\textit{CS 6390 - Information Extraction}}} % Institution

\date{\today} % Date

%----------------------------------------------------------------------------------------

\begin{document}

\maketitle % Print the title section

%----------------------------------------------------------------------------------------
%	ABSTRACT AND KEYWORDS
%----------------------------------------------------------------------------------------

%\renewcommand{\abstractname}{Summary} % Uncomment to change the name of the abstract to something else


%----------------------------------------------------------------------------------------
%	ESSAY BODY
%---------------------------------------------------------------------------------------
 
%------------------------------------------------

\section*{Evaluation Results}

Below is a table of the accuracy and F1-score for various systems and benchmarks. To evaluate my system, I used only one of the sixty categories of documents and Q/A's for time reasons (would take multiple days train a system across all the training examples).

\begin{table}[H]
\centering
{\renewcommand{\arraystretch}{1.2}%
\begin{tabular}{| c | c | c | c |}
\hline
Approach & Dataset & Accuracy & F1-Score\\
\hline
Random Guess & Dev & 1.1\% & 4.1\%\\ \hline
My Basic System  & Dev & 5.8\% & 10.7\%\\ \hline
\textbf{My Final System} & \textbf{Dev} & \textbf{14.57\%} & \textbf{21.64\%} \\ \hline
State-of-the-art & Test & 76.92\% & 84\% \\ \hline
Human Baseline & Dev & 80.3\% & 90.5\%\\ \hline
\end{tabular}}
\caption{Bold indicating my current system. Approaches in ascending order based on performance.}
\end{table}

Additionally I wanted to evaluate how well the heuristic of using the noun phrases in the document as the candidate pool of answers for a question. I took 2,000 questions and computed how often one of the possible answers to a question ended up in the candidate pool.

\begin{table}[H]
\centering
{\renewcommand{\arraystretch}{1.2}%
\begin{tabular}{| c | c | c |}
\hline
Questions & Answers Found & Percent\\
\hline
2,000 & 1,105 & 55.25\%  \\ \hline
\end{tabular}}
\end{table}

This approach puts a low ceiling on the possible recall of my system. A better approach to finding candidates would be worth exploring.

\end{document}